\documentclass[10pt, twoside]{article}
\usepackage{fancyhdr}
\usepackage{amsmath, amsthm, amssymb}
\usepackage[catalan]{babel}
\usepackage[titles]{tocloft}
\usepackage[utf8]{inputenc}
\usepackage[left=2.15cm, right=2.15cm, top=30mm, bottom=20mm]{geometry}
\usepackage{parskip}
\setlength{\parindent}{0pt} % Elimina la sangría de la primera línea de cada párrafo
\usepackage{titlesec}
\usepackage{bookmark}
\usepackage{multirow}
\usepackage{graphicx}
\usepackage{physics}
\usepackage{hyperref}
\usepackage{float}
\usepackage{caption}

\captionsetup{labelfont=bf}


\begin{document}

\begin{titlepage}
\centering
{\Large Fonaments d'Enginyeria Química \\ MO70399 \par}
\vspace{2cm}
{\Huge \textbf{Pràctica 2:} \par}
\vspace{1cm}
{\Huge \textbf{Balanç d'Energia Calorífica} \par}
\vspace{2cm}
{\Large Grup B \par}
\vspace{0.5cm}
{\Large Torn 2 \par}
\vspace{0.5cm}
{\normalsize Baldi Garcia, Isaac: 1667260 \\ Barbens Calzadilla, Carla: 1666167 \\ Belmonte Leiva, Marc: 1619451 \\ Bujones Umbert, Jun Shan: 1549086 \\ Franco Avilés, Eric: 1666739 \\ Gómez Rubio, Miquel: 1668850 \\ González Barea, Eric: 1672980 \\ Jacas García, Eira: 1666616 I NOMBRE DE PÀGINES AAAAAA\par}
\vspace{2cm}
{\Large Gener 2025 \par}
\vspace{2cm}
\includegraphics[width=0.4\textwidth]{Logo_UAB.png}


\end{titlepage}

\pagenumbering{gobble}
\renewcommand{\cftsecfont}{}
\renewcommand{\cftsecpagefont}{}
\renewcommand{\cftsecleader}{\cftdotfill{\cftdotsep}}
\renewcommand{\cftdotsep}{0.2}
\setlength{\cftbeforesecskip}{0.5em}
\setlength{\cftbeforesubsecskip}{0.5em}
\tableofcontents

\newpage
\pagenumbering{arabic}
\setcounter{page}{1}

\pagestyle{fancy}
\lhead{\textbf{Pràctica 2: Balanç d'energia calorífica}}
\rhead{\textbf{Fonaments d'Enginyeria Química}}

\begin{abstract}
En aquesta pràctica ens proposem estudiar els balaços d'energia calorífica aplicats tanc adiabàtic, en el qual no es produeix cap tipus d'intercanvi d'energia i/o matèria, i en concret de calor, amb l'entorn.  Per tal de demostrar experimentalment això, mesurarem la temperatura de l'aigua que flueix per dins del reactor en diferents temps, comparant-los amb la temperatura del tanc pulmó.
\end{abstract}

\section{Resultats i discussió}
Abans de començar amb la part experimental cal que, prèviament, calibrem la bomba, per tal de conèixer quins cabals es corresponen amb cada valor de rpm's de la bomba, i mesurem el volum del tanc.

\subsection{Calibratge de la bomba d'entrada}
Per calibrar la bomba hem fet un seguit de mesures dels cabals corresponents als valors de revolucions per minut (rpm) donats per la bomba, tot calculant la quanitat de volum que surt en un temps de 3 minuts
\begin{equation}
    Q_L = \frac{V}{t}
\end{equation}

\subsection{Mesura del volum del tanc}

\section{Conclusions}




\section{Calibratge}


\end{document}