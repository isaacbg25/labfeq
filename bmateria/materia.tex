\documentclass[10pt, twoside]{article}
\usepackage{fancyhdr}
\usepackage{amsmath, amsthm, amssymb}
\usepackage[catalan]{babel}
\usepackage[titles]{tocloft}
\usepackage[utf8]{inputenc}
\usepackage[left=2.15cm, right=2.15cm, top=30mm, bottom=20mm]{geometry}
\usepackage{parskip}
\setlength{\parindent}{0pt} % Elimina la sangría de la primera línea de cada párrafo
\usepackage{titlesec}
\usepackage{hyperref}
\usepackage{float}
\usepackage{caption}
\usepackage{graphicx} % Required for inserting images
\usepackage[T1]{fontenc}
\usepackage{hyperref}
\usepackage{multirow}
\usepackage{subcaption}
\usepackage{booktabs}
\usepackage{bookmark}
\usepackage{graphicx}
\usepackage{setspace}
\usepackage{physics}
\usepackage{tikz}
\usepackage{tikz-3dplot}
\usepackage[outline]{contour} % glow around text
\usepackage{xcolor}
\usepackage{makeidx}
\usepackage{pgfplots}
\pgfplotsset{compat=1.18}
\usepackage{caption}
\setlength{\parskip}{11pt}
\usepackage{listings}
\raggedbottom

\captionsetup{labelfont=bf}

\begin{document}

\begin{titlepage}
\centering
{\Large Fonaments d'Enginyeria Química \\ MO70399 \par}
\vspace{2cm}
{\Huge \textbf{Pràctica 1:} \par}
\vspace{1cm}
{\Huge \textbf{Balanç de matèria} \par}
\vspace{1cm}
{\Large Grup B \par}
\vspace{0.5cm}
{\Large Torn 2 \par}
\vspace{0.5cm}
{\normalsize Baldi Garcia, Isaac: 1667260 \\ Barbens Calzadilla, Carla: 1666167 \\ Belmonte Leiva, Marc: 1619451 \\ Bujones Umbert, Jun Shan: 1549086 \\ Franco Avilés, Eric: 1666739 \\ Gómez Rubio, Miquel: 1668850 \\ González Barea, Eric: 1672980 \\ Jacas García, Eira: 1666616 \par}
\vspace{1cm}
{\normalsize 6 pàgines \par}
{\Large Gener 2025 \par}
\vspace{2cm}
\includegraphics[width=0.4\textwidth]{Logo_UAB.png}

\end{titlepage}

\pagenumbering{gobble}
\renewcommand{\cftsecfont}{}
\renewcommand{\cftsecpagefont}{}
\renewcommand{\cftsecleader}{\cftdotfill{\cftdotsep}}
\renewcommand{\cftdotsep}{0.2}
\setlength{\cftbeforesecskip}{0.5em}
\setlength{\cftbeforesubsecskip}{0.5em}
\tableofcontents

\newpage
\pagenumbering{arabic}
\setcounter{page}{1}

\pagestyle{fancy}
\lhead{\textbf{Pràctica 1: Balanç de matèria}}
\rhead{\textbf{Fonaments d'Enginyeria Química}}

\begin{abstract}
En aquesta pràctica, el nostre objectiu era aplicar el balanç de matèria a un reactor de tanc agitat per on circula aigua mantenint el volum constant. Primerament, hem hagut de construir dues rectes de calibratge per poder relacionar les mesures insturmentals amb les dades que ens interessàva estudiar. Segonament, hem muntat un sistema en què podiem mesurar la variació de la concentració de sal d'una dissolució aquosa en el reactor en operació en continu. Així hem pogut comparar els resultats teòrics amb els experimentals.
\end{abstract}

\section{Calibratge}
Per dur a terme l'experiment primer hem calibrat els instruments que ho necessitaven, en aquest cas, la bomba i el conductímetre. 

\subsection{Recta de calibratge: concentració sal - conductivitat dissolució}


\begin{figure}[hbt!]
    \centering
    % Primera part: Imatge en una minipage
    \begin{minipage}{0.45\textwidth}
        \centering
        \includegraphics[width=0.9\linewidth]{calisal.png} % Substitueix per la teva imatge
        \caption{Recta de calibratge del Conductímetre}
        \label{fig:calisal}
    \end{minipage}%
    \hfill 
    \begin{minipage}{0.45\textwidth}
        \centering
        \captionof{table}{Calibratge del conductímetre}
        \label{tab:calisal}
        \begin{tabular}{|l|c|r|}
            \hline
            Conductivitat (mS)	&	Massa sal (g)	\\ \hline
            0,626	&	0,000	\\ \hline
            13,87	&	0,700	\\ \hline
            25,7	&	1,410	\\ \hline
            35,8	&	2,118	\\ \hline
            47,2	&	2,810	\\ \hline
            56,1	&	3,502	\\ \hline
            65,5	&	4,213	\\ \hline
                          
        \end{tabular}
    \end{minipage}

\end{figure}



\subsection{Corba de calibratge: cabal volumètric - revolucions per minut bomba}

\begin{figure}[hbt!]
    \centering
    % Primera part: Imatge en una minipage
    \begin{minipage}{0.45\textwidth}
        \centering
        \includegraphics[width=0.9\linewidth]{calibomba.png} % Substitueix per la teva imatge
        \caption{Recta de calibratge de la bomba}
        \label{fig:calibomba}
    \end{minipage}%
    \hfill 
    \begin{minipage}{0.45\textwidth}
        \centering
        \captionof{table}{Calibratge de la Bomba}
        \label{tab:calibomba}
        \begin{tabular}{|l|c|r|}
            \hline
            rpm	&	Cabal (mL/min)	\\ \hline
            150	&	566,6666667	\\ \hline
            140	&	516,6666667	\\ \hline
            135	&	516,6666667	\\ \hline
            130	&	500	\\ \hline
            125	&	483,3333333	\\ \hline
            100	&	400	\\ \hline
            70	&	275	\\ \hline
            75	&	300	\\ \hline
                          
        \end{tabular}
    \end{minipage}

\end{figure}


Usant el programari Gnuplot$\copyright$ hem construit aquests gràfics i n'hem generat les equacions de la recta de regressió.
Aquesta recta ens servira en el primer cas per ajustar el cabal a partir de les revolucions de la bomba i en el segon cas per trobar els valors de la concentració de sal dins del tanc a partir de les dades conductiomètriques

\section{Resultats experimentals}

\subsection{Volum del tanc}
Hem mesurat el volum del tanc: $\fbox{V=4L}$

Massa sal perquè concentració dins tanc sigui 40g/L: \fbox{$m_{sal} = 160g$}

\subsection{Temps de residència}
Temps de residència cabal=270.16 mL/min calculat: \fbox{$\tau= 888.38s$}

Temps de residència cabal=491.42 mL/min calculat: \fbox{$\tau = 488.39s$}

Temps de residència obtingut experimentalment pel cabal=270.16 mL/min : \fbox{$\tau_{regr}=783.00s$}

Temps de residència obtingut experimentalment pel cabal=491.42 mL/min : \fbox{$\tau_{regr}=448.67s$}

Error relatiu de $\tau_{Q_L = 270.16 mL/min} = \fbox{11,86 \%}$

Error relatiu de $\tau_{Q_L = 491.42 mL/min} = \fbox{8,132 \%}$

\subsection{Concentració del cabal de sortida}

\begin{figure}[hbt!]
    \centering
    \begin{subfigure}{0.45\textwidth}
        \centering
        \includegraphics[width=\textwidth]{conc300.png}
        \caption{Concentració al llarg del temps per al cabal de 300 L/min (exp:270 L/min).}
        \label{fig:conc300}
    \end{subfigure}
    \hspace{0.025\textwidth}
    \begin{subfigure}{0.45\textwidth}
        \centering
        \includegraphics[width=\textwidth]{conc500.png}
        \caption{Concentració al llarg del temps per al cabal de 500 L/min (exp:491 L/min).}
        \label{fig:conc500}
    \end{subfigure}
    \caption{Concentració cabal de sortida}
    \label{fig:concs}
\end{figure}

A $t=3\tau$ obtenim les següents dades teòriques i experimentals.

\begin{minipage}{\textwidth}
    \captionsetup{hypcap=false}
    \centering
    \captionof{table}{Concentracions a temps $3\tau$}
    \label{tab: conce3tau}
    \begin{tabular}{|l|c|c|r|}
        \hline
        cabal       &	$C_{teo}$ (g/L)	&	$C_{exp}$ (g/L)	    &   Error relatiu	\\ \hline
        cabal=500	&	2,837248504	&	1,73982068	&	38,68$\%$	\\ \hline
        cabal=300	&	2,078515395	&	1,315083715	&	36,73$\%$	\\ \hline
    \end{tabular}
\end{minipage}

\pagebreak
\begin{figure}[h!]
    \centering
    \begin{subfigure}{0.45\textwidth}
        \centering
        \includegraphics[width=\textwidth]{error300.png}
        \caption{Error relatiu per al cabal de 300 L/min (exp: 270.16 L/min).}
        \label{fig: error300}
    \end{subfigure}
    \hspace{0.025\textwidth}
    \begin{subfigure}{0.45\textwidth}
        \centering
        \includegraphics[width=\textwidth]{error500.png}
        \caption{Error relatiu per al cabal de 500 L/min (exp: 491.42 L/min).}
        \label{fig: error500}
    \end{subfigure}
    \caption{Error relatiu concentració cabal de sortida}
    \label{fig: errors}
\end{figure}

A la figura \ref{fig: error300} observem que l'error incrementa exponencialment amb el temps. Aquesta és la tendència que s'observa quan els errors són deguts a alguna aproximació teòrica. En canvi a la figura \ref{fig: error500} observem l'efecte combinat entre aquest tipus d'error i l'originat per l'operari.

\section{Conclusions}
Amb aquesta pràctica hem pogut comprovar que l'equació del balanç de matèria ens serveix per descriure el que ocórre en un reactor de tanc agitat en règim transitori. Com dicta aquest balanç hem vist que el cabal és invèrsament proporcional al temps de residència i, per tant, directament proporcional a la disminució de sal dins del reactor.

El model teòric és consistent amb els resultats experimentals com podem veure en les diverses gràfiques. Tot i així sí que hi ha una certa discrepància entre els valors que podrien ser causa de les aproximacions i els errors experimentals. 

\pagebreak
\section*{Annex}

\appendix

\section{Hipòtesis i càlculs}

En aquesta pràctica hem considerat que el reactor de tanc agitat és ideal i, per tant, la mescla dins del tanc és perfecta. Això vol dir que podem assumir que la concentració de dins del tanc és la mateixa que la del cabal de sortida. Per aquesta raó, mesurem la conductiivitat de la dissolució quan encara està en el tanc.

També hem considerat que la concentració de sal de l'aigua de l'aixeta és tan petita que la podem despreciar. Per tant $c_{ref} = 0$.

Amb el volum del reactor ja podem calcular la quantitat de sal que necessitem perquè la concentració inicial dins del tanc sigui de $40g/L$. 
    $m_{sal} = V_{tanc}\frac{40g sal}{1L} = \fbox{160g sal}$
Amb el volum del tanc també podem calcular el temps de residència teòric per cada cabal.
    $\tau = V\frac{V}{Q_L}$

Calculem la concentració per a cada temps mitjançant l'equació proporcionada al guió:
\begin{equation}
    c(t) = c_{j_1} + (c_o-c_{j_1})\exp\left(-\frac{Q_L}{V}t\right) 
    \label{c(t)}   
\end{equation}

on $c_{j_1}$ és la concentració del cabal d'entrada. En aquest cas, és la concentració de sal que té la propia aigua d'aixeta (que hem considerat nul·la) i $c_o$ la concentració inicial de sal dins del tanc.
Grafic 1(teoric + exp cabal1), Gràfic2 (teo+exp cabal2)

Linealitzant l'equació \eqref{c(t)} fem el gràfic $\log(c'(t))$ enfront $t$ i evaluant-ne el pendent obtenim el temps de residència fent un petit càlcul.

$\tau_{regr} = \frac{-1}{2.303m}$ on $m$ és el pendent de la recta de regressió.

Estudiem les diferències calculant l'error relatiu de les concentracions i del temps de residència, expressem el resultats en percentatges.

Error  relatiu  de  $c(t) = \frac{|c(t)_{teo} - c(t)_{exp} exp|}{c(t)_{teo}} 100$

Error relatiu de $\tau_{Q_L} = \frac{|\tau_{teo} - \tau_{exp}|}{\tau_{teo}} 100$

\section{Taules dades experimentals}

\begin{minipage}{\textwidth}
    \centering
    \captionsetup{hypcap=false}
    \captionof{table}{Concentracions al llarg del temps pel cabal de 270.16 mL/min}
   
    \label{tab:taulaconc300}
    \begin{tabular}{|l|c|c|c|c|r|}
        \hline
        Temps (s)	&	Conductivitat (mS)	&	Concentració (g/L)	&	Temps (s)	&	Conductivitat (mS)	&	Concentració (g/L)	\\ \hline
        0	&	63,2	&	39,5	&	600	&	33	&	19,8	\\ \hline
        15	&	62,9	&	39,3	&	660	&	31,5	&	18,8	\\ \hline
        30	&	61,9	&	38,7	&	720	&	29,8	&	17,7	\\ \hline
        45	&	60,9	&	38,0	&	780	&	27,9	&	16,5	\\ \hline
        60	&	59,9	&	37,4	&	840	&	26,1	&	15,3	\\ \hline
        75	&	59	&	36,8	&	900	&	24,5	&	14,2	\\ \hline
        90	&	57,9	&	36,1	&	960	&	23,1	&	13,3	\\ \hline
        105	&	57,3	&	35,7	&	1020	&	21,6	&	12,3	\\ \hline
        120	&	56,3	&	35,0	&	1080	&	20,3	&	11,5	\\ \hline
        135	&	55,3	&	34,4	&	1140	&	19,04	&	10,7	\\ \hline
        150	&	54,5	&	33,8	&	1200	&	17,84	&	9,9	\\ \hline
        165	&	53,7	&	33,3	&	1260	&	16,73	&	9,2	\\ \hline
        180	&	52,9	&	32,8	&	1320	&	15,72	&	8,5	\\ \hline
        195	&	52	&	32,2	&	1380	&	14,69	&	7,8	\\ \hline
        210	&	51,1	&	31,6	&	1440	&	13,79	&	7,2	\\ \hline
        225	&	50,5	&	31,2	&	1500	&	12,96	&	6,7	\\ \hline
        240	&	49,5	&	30,6	&	1560	&	12,13	&	6,2	\\ \hline
        255	&	48,9	&	30,2	&	1620	&	11,45	&	5,7	\\ \hline
        270	&	48,1	&	29,7	&	1680	&	10,72	&	5,2	\\ \hline
        285	&	47,2	&	29,1	&	1740	&	10,05	&	4,8	\\ \hline
        300	&	46,5	&	28,6	&	1800	&	9,3	&	4,3	\\ \hline
        330	&	45,2	&	27,8	&	1860	&	8,88	&	4,0	\\ \hline
        360	&	43,7	&	26,8	&	1920	&	8,32	&	3,7	\\ \hline
        390	&	42,4	&	25,9	&	1980	&	7,72	&	3,3	\\ \hline
        420	&	41	&	25,0	&	2040	&	7,34	&	3,0	\\ \hline
        450	&	39,7	&	24,2	&	2100	&	6,9	&	2,7	\\ \hline
        480	&	38,3	&	23,3	&	2160	&	6,49	&	2,5	\\ \hline
        510	&	37,2	&	22,5	&	2220	&	6,09	&	2,2	\\ \hline
        540	&	36,1	&	21,8	&	2280	&	5,73	&	2,0	\\ \hline
        570	&	34,7	&	20,9	&	2340	&	5,38	&	1,7	\\ \hline
    \end{tabular}
\end{minipage}

\begin{minipage}{\textwidth}
    \centering
    \captionsetup{hypcap=false}
    \captionof{table}{Concentracions al llarg del temps pel cabal de 491.42 mL/min}
   
    \label{tab:taulaconc500}
    \begin{tabular}{|l|c|c|c|c|r|}
        \hline
        Temps (s)	&	Senyal (mS)	&	Concentració (g/L)	&	Temps (s)	&	Senyal (mS)	&	Concentració (g/L)	\\ \hline
        0	&	63,4	&	39,65249559	&	390	&	33,2	&	19,91856277	\\ \hline
        15	&	61,2	&	38,21492432	&	420	&	31,4	&	18,7423681	\\ \hline
        30	&	59,6	&	37,16941795	&	450	&	29,7	&	17,63151757	\\ \hline
        45	&	58	&	36,12391157	&	480	&	28,1	&	16,5860112	\\ \hline
        60	&	56,5	&	35,14374935	&	510	&	26,6	&	15,60584897	\\ \hline
        75	&	54,4	&	33,77152223	&	540	&	25,1	&	14,62568675	\\ \hline
        90	&	53,5	&	33,1834249	&	570	&	23,8	&	13,77621282	\\ \hline
        105	&	52	&	32,20326267	&	600	&	22	&	12,60001815	\\ \hline
        120	&	50,7	&	31,35378874	&	660	&	20,1	&	11,35847933	\\ \hline
        135	&	49,6	&	30,63500311	&	720	&	17,86	&	9,894770402	\\ \hline
        150	&	48,9	&	30,17759407	&	780	&	16,02	&	8,692438071	\\ \hline
        165	&	48,6	&	29,98156162	&	840	&	14,19	&	7,496640155	\\ \hline
        180	&	48,4	&	29,85087333	&	900	&	12,8	&	6,588356492	\\ \hline
        195	&	47,6	&	29,32812014	&	960	&	11,44	&	5,699676074	\\ \hline
        210	&	46,1	&	28,34795791	&	1020	&	10,24	&	4,915546293	\\ \hline
        225	&	45	&	27,62917228	&	1080	&	9,12	&	4,18369183	\\ \hline
        240	&	43,8	&	26,8450425	&	1140	&	8,21	&	3,58906008	\\ \hline
        255	&	42,5	&	25,99556857	&	1200	&	7,33	&	3,014031574	\\ \hline
        270	&	41,4	&	25,27678294	&	1260	&	6,57	&	2,517416046	\\ \hline
        285	&	40,3	&	24,55799731	&	1320	&	5,88	&	2,066541422	\\ \hline
        300	&	39,3	&	23,90455582	&	1380	&	5,27	&	1,667942116	\\ \hline
        330	&	37,1	&	22,46698456	&	1440	&	4,73	&	1,315083715	\\ \hline
        360	&	35,1	&	21,16010159	&		&		&		\\ \hline				
    \end{tabular}
\end{minipage}

\end{document}